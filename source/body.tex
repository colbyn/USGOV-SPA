
\section{Meta}

For this assignment, there was an additional option for Utah County residents to chose Proposition 9, which is my topic of choice for this assignment. 


\section{Introduction — Proposed Amendment}

Proponents of Proposition 9 seek to reform the regional government from it’s current system consisting of three full-time commissioners (all three of whom hold executive and legislative powers) into something where residents,
\begin{quotation}
    ``would be better served by a government that includes a separation of powers, checks and balances, and regional representation'' \textsuperscript{†} \textsuperscript{(see the first essay)}
\end{quotation}

On the contrary, perhaps it could be argued that any system of ``checks and balances'' will -generally speaking- be less efficient than those where power is generally unopposed, and in this regard, some opponents of proposition 9 claim that the current system is comparatively efficient.\textsuperscript{†}

Yet the current system of three full-time commissioners depends on $\frac{2}{3}$ consensus, and if such commissioners are elected from constituents with opposing needs, the system will perhaps be just as inert.

Opponents claim that the current system is efficient, whereas the system proposed in Proposition 9 will entail a higher tax burden on local residents, and more bureaucracy, claiming that, 
\begin{quotation}
    ``Proposition 9 is not about improving government—it’s about expanding government on the backs of Utah County taxpayers.'' \textsuperscript{†} 
\end{quotation}

With an added note that: ``With the economy down and continued economic uncertainty under COVID-19, now is a terrible time to consider growing county government.''\textsuperscript{†} 

\section{Possible Outcomes}

Proposition 9 will entail a single person acting as mayor, in conjunction with five part-time council members representing the diverse geographical areas of the Utah County region. 

Proponents say that such a system will better accommodate the diverse needs of the Utah County region VIA a more decentralized form of government. With the current “three commissioner” based system, if all three of whom are from, say Provo, this provides Provo with considerable benefits, compared to some rural town with with a thousand or so members. Furthermore as things stand, given that the three commissioners are elected by popular vote, they have little or no incentive to consider the needs of such towns, likewise when push comes to shove, there are no incentives to compromise with such insignificant towns. 


\section{Regarding groups involved in the development of the proposed amendment}

I cannot find any sources documenting the development chronology of Proposition 9 explicitly.


Interestingly, an organization known as the ``Utah County Good Governance Advisory Board'' (an organization seemingly without an official website), commissioned Luke Peterson, along with a group of students from all Utah Valley University, for the purpose of researching forms of county governance. In this work, they ``analyzed governance practices across the more than 3,000 counties'', and therein, they ``identified 50 counties that were comparable to Utah County'', thereafter as written in the Deseret News, ``Among the 50 comparable counties, Utah County was the only county still operating under the three commissioner form of government.''\textsuperscript{‡}

The aforementioned conclusion is interesting, as though the currently three commissioner based system is a historical artifact, that historical precedence suggests will be phased out. So therefore, perhaps we can conclude that the development of Proposition 9 will have manifested -in some form or fashion- regardless of any single organization or party, if the needs of such regions demand for such.

Although I can imagine local businesses generally favoring Proposition 9, on the grounds that checks and balances in more conducive to stability compared to the whims of three persons with executive and legislative powers. But the aforementioned Luke Peterson remarked in the same article that the current three commissioner based system is ``mired in scandal''. But I digress.


\section{Aftermath}

Proposition 9 was defeated\textsuperscript{*}, therefore -for the time being- Utah County remains a three commissioner based system. Although I can imagine the momentum behind Proposition 9 will remain, since (as mentioned previously) some proponents claim that the three commissioner based system is a historical artifact, and therein most districts based on such will eventually transition to something akin what we see in Proposition 9.\textsuperscript{‡}
